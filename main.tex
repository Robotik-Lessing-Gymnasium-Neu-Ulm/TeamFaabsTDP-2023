\documentclass{scrartcl}

%Packages
\usepackage[utf8]{inputenc}
\usepackage[english]{babel}
\usepackage{graphicx}
\usepackage{wrapfig}

\title{Team Faabs - Technical Paper}
\author{Fabian Brune, Mark Krause, Jurij Lenz}
\date{May 2023}


\begin{document}


    \maketitle
    %\begin{center}

       % \includegraphics[width=\textwidth]{img/}
       % \caption{Robots-2023}
       % \label{fig:Robots}
    %\end{figure}
    \newpage


    \tableofcontents
     \newpage


     %Team general
    \section{introduction}
    \subsection{Team}

    %\begin{figure}[h]
        %\centering
        %\includegraphics[width=\textwidth]{img/}
        %\caption{Team Faabs - Jurij Lenz, Fabian Brune, Mark Krause}
        %\label{fig:team}
    %\end{figure}
    In our Team, everyone has a specific task to do.
\begin{enumerate}
    \item{Mark Krause: Software $\rightarrow$ }
    \item{Fabian Brune: Hardware  $\rightarrow$ }
    \item{Jurij Lenz: Raw Hardware $\rightarrow$ }
\end{enumerate}
\newpage

subsection{School}
%\begin{figure}[ht]
   % \centering
   % \includegraphics[width=\textwidth]{img/LGNU.jpg}
    %\caption{Lessing Gymnasium Neu-Ulm - German Open}
   % \label{fig:LGNU}
%\end{figure}

The Robotics program of our school was created 2011. Since then, we managed to win the World Open
multiple times in either Soccer LightWeight, Soccer Open or OnStage.
We try to put our new teams as fast as possible in the top leagues. With the gathered experience, we always
have teams to follow our footsteps.
\newline
\subsection{Abstract}
We are a Team of three students from the Lessing Gymnasium Neu-Ulm in Germany. Furthermore, we founded our Team
in 2019 and first participated in the RoboCup Junior in 2019. Robotic is a big part of our daily life.
We meet on school days and even on weekends and holidays.
\newline
\newline
We started developing our Robots in mid 2022 and had a first Prototype in late 2022. After final design
choices, we had our Robots for the German South-Open in early 2023. From this point on, some progress was
made in the Hardware sector.
After optimizing our program, we managed to reach the second place.
\newline
After the South-Open we redesigned our kicker and continued fixing small hard-\& software problems. We discovered, that with two cameras
instead of using a bad dribbler, just to dont use one at the German Open, because it worsend our movement with the ball .
With these improvements we managed to gather the second place at the German Open and qualified for WM.

\section{Development \& Testing}

\subsection{Development}
Due to the pandemic we weren't able to get much experience. In 2022 we had our first real RoboCup
in the LightWeight International League where we got second at the EM. With this experience we started to design the 2023 Open Robot.
\newline
The design of our robots aims to be as rigid as possible, while keeping the robot manouverable.
To achieve this goal, we used metal and 3D printed Parts in combination with optimised circuit boards.
In combination with our Software, we are able to travel with high speeds to the ball and
hit a goal without going out of bounce (most of the time).

\subsection{Mechanical Design}
All our parts were designed using 'Autodesk Inventor 2023 Professional' and 'Autodesk Eagle 9.6.2'.
Most of our parts are printed with our Prusa MK3S+ and the metal parts are produced
by our sponsor "Belch und Techhnik".
\newline
\newline

%\begin{wrapfigure}{r}{4cm}
  %  \centering
   % \includegraphics[width=1\linewidth]{img/inv/OmniWheel.jpg}
   % \caption{Omniwheels}
    %\label{fig:OmniWheels}
%\end{wrapfigure}

The whole design aims to be fast, reliable and robust. To achieve this goal,
 we put all the force in a powerfull desing.
\newline
Traveling with high speeds, while being strong is also a huge challenge. To achieve this, we use high quality
Maxon DC Motors in combination with the VNH3SP30 driver chip. To put the force on the ground, we designed
new Omniwheels%~\ref{fig:OmniWheels}.
The OmniWheel consists of two independent aluminum pieces screwed together. For the small wheels we use a small
aluminum piece covered with an O-Ring out of EPDM plastic.
\newline
To protect this whole construction from the opponent robot, everything is moved into the robot and a 3D
printed and metal protectors are covering the whole inner side of the robot.
\newpage


\subsection{Electrical Design}
To detect various elements and control the whole robot we use a Teensy 4.1 microcontroller and a jetson nano
for ball detection. All electronic parts are soldered to circuit boards.
\newline
We have a total of four Circuit Boards:
\begin{enumerate}
    \item{Line Detection\& motordriver PCB}
    \item{camera\&lidar PCB}
    \item{Controller PCB}
\end{enumerate}
Each Circuit Board has a predetermined task and all the above are controlled by the Teensy 4.1 
and the Jetson Nano 2Gb.

We order our Circuit Boards from a local company and solder the parts manually on the boards.

\subsubsection{Line Detection\& motordriver PCB}

%\begin{wrapfigure}{r}{4cm}
   % \centering
   % \includegraphics[width=0.75\linewidth]{img/eagle/LineDedectionPCB.png}
   % \caption{PCB}
    %\label{fig:LDPCB}
%\end{wrapfigure}

The Line Detection PCB consists of a circle with a cross of phototransistor pairs, our motordrivers and the kicker control . To connect all
48 sensors to our main controller, we use analog multiplexer. The multiplexers are switched parallel
to save ports.
\newline
As you can see on the right, we use pairs of two phototransistors and one led. For us, this is the 
best way to detect the line.

\subsubsection*{camera\&lidar PCB}
On this very small and special PCB, we combine our most important sensor, the lidars~\ref{lidars} and our raspery pie 
pie cam. The camera is used for ball and goal detection. It is directly programmed on our Jetson in OpenCv.
Our lidars are conected to the Teensy 4.1 on the Controller~\label{PCB:Controller} PCB.

\subsubsection{Controller PCB}

On our controller PCB we got the most imortant parts of our Robot like the Nvidia Jetson Nano 2gb 
or the Teensy 4.1. Our dribbler is also located on our main PCB. We use our dribbling device
to give the ball a backspin, so that it stays right infront of our kicker. You can also find our baterry pack 
and a diod. 

 \subsection{Software} %hier muss noch UNDBEDINGT, zwischen Torwart und Stürmer unterschieden werden.
 As we all know, that's the objective:
 \begin{enumerate}
     \item{Approach the ball quickly and precisely to get it into the ball pit.}
     \item{Aim for the goal as quickly as possible, while ball is save.}
     \item{Score as fast as possible.}
 \end{enumerate}
 Accordingly, a modularization of the code can be set up.
 % Das muss mArk schreiben!!
 \subsubsection{Ball approach}
 
 \subsubsection{Aiming at the goal}
 
 \subsubsection{Shot}
 
 \section{Experience 2023}
 



\end{document}



